% ==============================================================================

\documentclass[useAMS,usenatbib]{mn2e}

\usepackage[dvips]{graphicx}   % This is necessary to be able to include graphic
\usepackage{times}

\usepackage{epsfig}
\usepackage{amsmath,amssymb}
\usepackage{latexsym}
\usepackage{graphicx}% Include figure files
\usepackage{dcolumn}% Align table columns on decimal point
\usepackage{bm}% bold math
%\usepackage{epstopdf}

\def\reff@jnl#1{{\rm#1\/}}
\def\aj{\reff@jnl{AJ}}                  % Astronomical Journal
\def\araa{\reff@jnl{ARA\&A}}            % Annual Review of Astron and Astrophys
\def\apj{\reff@jnl{ApJ}}                % Astrophysical Journal
\def\apjl{\reff@jnl{ApJ}}               % Astrophysical Journal, Letters
\def\apjs{\reff@jnl{ApJS}}              % Astrophysical Journal, Supplement
\def\ao{\reff@jnl{Appl.Optics}}         % Applied Optics
\def\apss{\reff@jnl{Ap\&SS}}            % Astrophysics and Space Science
\def\aap{\reff@jnl{A\&A}}               % Astronomy and Astrophysics
\def\aapr{\reff@jnl{A\&A~Rev.}}         % Astronomy and Astrophysics Reviews
\def\aaps{\reff@jnl{A\&AS}}             % Astronomy and Astrophysics, Supplement
\def\azh{\reff@jnl{AZh}}                % Astronomicheskii Zhurnal
\def\baas{\reff@jnl{BAAS}}              % Bulletin of the AAS
\def\jrasc{\reff@jnl{JRASC}}            % Journal of the RAS of Canada
\def\memras{\reff@jnl{MmRAS}}           % Memoirs of the RAS
\def\mnras{\reff@jnl{MNRAS}}            % Monthly Notices of the RAS
\def\pra{\reff@jnl{Phys.Rev.A}}         % Physical Review A: General Physics
\def\prb{\reff@jnl{Phys.Rev.B}}         % Physical Review B: Solid State
\def\prc{\reff@jnl{Phys.Rev.C}}         % Physical Review C
\def\prd{\reff@jnl{Phys.Rev.D}}         % Physical Review D
\def\prl{\reff@jnl{Phys.Rev.Lett}}      % Physical Review Letters
\def\pasp{\reff@jnl{PASP}}              % Publications of the ASP
\def\pasj{\reff@jnl{PASJ}}              % Publications of the ASJ
\def\qjras{\reff@jnl{QJRAS}}            % Quarterly Journal of the RAS
\def\skytel{\reff@jnl{S\&T}}            % Sky and Telescope
\def\solphys{\reff@jnl{Solar~Phys.}}    % Solar Physics
\def\sovast{\reff@jnl{Soviet~Ast.}}     % Soviet Astronomy
\def\ssr{\reff@jnl{Space~Sci.Rev.}}     % Space Science Reviews
\def\zap{\reff@jnl{ZAp}}                % Zeitschrift fuer Astrophysik
\def\nat{\reff@jnl{Nature}}             % Nature 
\newcommand{\be}{\begin{equation}}
\newcommand{\ee}{\end{equation}}
\newcommand{\bea}{\begin{eqnarray}}
\newcommand{\eea}{\end{eqnarray}}
\newcommand{\bi}{\begin{itemize}}
\newcommand{\ei}{\end{itemize}}
% \newcommand\farcs{\mbox{$.\!\!^{\prime\prime}$}}
\def\vtr#1{{\bf #1}}
\def\mtx#1{{\bf #1}}

% ==============================================================================

\title%
[Astronomical Image Annotation]%
{AstroTaches: How Useful Are Annotative Paintings of Astronomical Images?}
\label{firstpage}

\author%
[Marshall et al]%
{Phil Marshall$^{1}$\thanks{E-mail:pjm@physics.ucsb.edu}, 
David W. Hogg$^{2}$,
Stuart R. Lowe$^{3}$
\newauthor Pamela L. Gay$^{4,5}$\\
$^{1}$Physics department, University of Oxford, \\
$^{2}$CCPP, NYU, New York, USA\\
$^{3}$Cardiff University, UK\\
$^{4}$Middle Illinois, USA\\
$^{5}$AstroSphere, USA}

\date{today}

\pagerange{\pageref{firstpage}--\pageref{lastpage}}

\pubyear{2011}

% ===============================================================================

\begin{document}

\maketitle

%-------------------------------------------------------------------------------

\begin{abstract}



\end{abstract}


\begin{keywords}
gravitational lensing --- surveys --- cosmology: observations
\end{keywords}


%-------------------------------------------------------------------------------

\section{Introduction}


%-------------------------------------------------------------------------------

\section{The AstroTaches Interface}
\label{sect:interface}



%-------------------------------------------------------------------------------

\section{A Simple Test Case: Gravitational Lens \lens}
\label{sect:testcase}


\begin{figure*}
\begin{center}
\includegraphics[width=0.9\linewidth]{figs/.eps} 
\caption{Predicted number density of lensing events $N$ per square degree as a function of the instrument angular resolution $\theta$/arcsec,
for the fixed values of 
    $\G=10^{-6}$ (dashed), 
       $10^{-7}$ (dotted) and
$5\times10^{-8}$ (solid).}\label{fig:bound}
\end{center}
\end{figure*}


%-------------------------------------------------------------------------------

\section{Paintings Generated by Zooniverse Users}
\label{sect:data}



%-------------------------------------------------------------------------------

\section{Analysis and Results}
\label{sect:results}



%-------------------------------------------------------------------------------

\section{Conclusions}
\label{sect:concl}

From our simple analysis we draw the following conclusions:

\begin{enumerate}

\item The paintings done by citizen scientists span ... 

\item The mean (stacked) image and corresponding uncertainty map ... 

\end{enumerate}

%-----------------------------------------------------------------------

\section*{Acknowledgments} 

Painting on images for feature identification 
was conceived in discussions with the Zooniverse team in Oxford,
including Chris Lintott, Aprajita Verma, Arfon Smith and Rob Simpson. The
AstroTaches project was begun at the Dot Astronomy 3 meeting; we are grateful
to New College, Oxford, for providing the venue, and the conference organisers
for allowing us to hack the project together in an afternoon. 
%
% PJM received support from the Royal Society in the form of a research
% fellowship.                

%-------------------------------------------------------------------------------
\label{lastpage}

% \bibliography{references}

% MNRAS can be tricked into accepting bibtex but I forget how... 
% I used bubble to make a bbl file from our bib file:

% bubble -f astrotaches.tex references.bib 

% ==============================================================================

\documentclass[useAMS,usenatbib]{mn2e}

\usepackage[dvips]{graphicx}   % This is necessary to be able to include graphic
\usepackage{times}

\usepackage{epsfig}
\usepackage{amsmath,amssymb}
\usepackage{latexsym}
\usepackage{graphicx}% Include figure files
\usepackage{dcolumn}% Align table columns on decimal point
\usepackage{bm}% bold math
%\usepackage{epstopdf}

\def\reff@jnl#1{{\rm#1\/}}
\def\aj{\reff@jnl{AJ}}                  % Astronomical Journal
\def\araa{\reff@jnl{ARA\&A}}            % Annual Review of Astron and Astrophys
\def\apj{\reff@jnl{ApJ}}                % Astrophysical Journal
\def\apjl{\reff@jnl{ApJ}}               % Astrophysical Journal, Letters
\def\apjs{\reff@jnl{ApJS}}              % Astrophysical Journal, Supplement
\def\ao{\reff@jnl{Appl.Optics}}         % Applied Optics
\def\apss{\reff@jnl{Ap\&SS}}            % Astrophysics and Space Science
\def\aap{\reff@jnl{A\&A}}               % Astronomy and Astrophysics
\def\aapr{\reff@jnl{A\&A~Rev.}}         % Astronomy and Astrophysics Reviews
\def\aaps{\reff@jnl{A\&AS}}             % Astronomy and Astrophysics, Supplement
\def\azh{\reff@jnl{AZh}}                % Astronomicheskii Zhurnal
\def\baas{\reff@jnl{BAAS}}              % Bulletin of the AAS
\def\jrasc{\reff@jnl{JRASC}}            % Journal of the RAS of Canada
\def\memras{\reff@jnl{MmRAS}}           % Memoirs of the RAS
\def\mnras{\reff@jnl{MNRAS}}            % Monthly Notices of the RAS
\def\pra{\reff@jnl{Phys.Rev.A}}         % Physical Review A: General Physics
\def\prb{\reff@jnl{Phys.Rev.B}}         % Physical Review B: Solid State
\def\prc{\reff@jnl{Phys.Rev.C}}         % Physical Review C
\def\prd{\reff@jnl{Phys.Rev.D}}         % Physical Review D
\def\prl{\reff@jnl{Phys.Rev.Lett}}      % Physical Review Letters
\def\pasp{\reff@jnl{PASP}}              % Publications of the ASP
\def\pasj{\reff@jnl{PASJ}}              % Publications of the ASJ
\def\qjras{\reff@jnl{QJRAS}}            % Quarterly Journal of the RAS
\def\skytel{\reff@jnl{S\&T}}            % Sky and Telescope
\def\solphys{\reff@jnl{Solar~Phys.}}    % Solar Physics
\def\sovast{\reff@jnl{Soviet~Ast.}}     % Soviet Astronomy
\def\ssr{\reff@jnl{Space~Sci.Rev.}}     % Space Science Reviews
\def\zap{\reff@jnl{ZAp}}                % Zeitschrift fuer Astrophysik
\def\nat{\reff@jnl{Nature}}             % Nature 
\newcommand{\be}{\begin{equation}}
\newcommand{\ee}{\end{equation}}
\newcommand{\bea}{\begin{eqnarray}}
\newcommand{\eea}{\end{eqnarray}}
\newcommand{\bi}{\begin{itemize}}
\newcommand{\ei}{\end{itemize}}
% \newcommand\farcs{\mbox{$.\!\!^{\prime\prime}$}}
\def\vtr#1{{\bf #1}}
\def\mtx#1{{\bf #1}}

% ==============================================================================

\title%
[Astronomical Image Annotation]%
{AstroTaches: How Useful Are Annotative Paintings of Astronomical Images?}
\label{firstpage}

\author%
[Marshall et al]%
{Phil Marshall$^{1}$\thanks{E-mail:pjm@physics.ucsb.edu}, 
David W. Hogg$^{2}$,
Stuart R. Lowe$^{3}$
\newauthor Pamela L. Gay$^{4,5}$\\
$^{1}$Physics department, University of Oxford, \\
$^{2}$CCPP, NYU, New York, USA\\
$^{3}$Cardiff University, UK\\
$^{4}$Middle Illinois, USA\\
$^{5}$AstroSphere, USA}

\date{today}

\pagerange{\pageref{firstpage}--\pageref{lastpage}}

\pubyear{2011}

% ===============================================================================

\begin{document}

\maketitle

%-------------------------------------------------------------------------------

\begin{abstract}



\end{abstract}


\begin{keywords}
gravitational lensing --- surveys --- cosmology: observations
\end{keywords}


%-------------------------------------------------------------------------------

\section{Introduction}


%-------------------------------------------------------------------------------

\section{The AstroTaches Interface}
\label{sect:interface}



%-------------------------------------------------------------------------------

\section{A Simple Test Case: Gravitational Lens \lens}
\label{sect:testcase}


\begin{figure*}
\begin{center}
\includegraphics[width=0.9\linewidth]{figs/.eps} 
\caption{Predicted number density of lensing events $N$ per square degree as a function of the instrument angular resolution $\theta$/arcsec,
for the fixed values of 
    $\G=10^{-6}$ (dashed), 
       $10^{-7}$ (dotted) and
$5\times10^{-8}$ (solid).}\label{fig:bound}
\end{center}
\end{figure*}


%-------------------------------------------------------------------------------

\section{Paintings Generated by Zooniverse Users}
\label{sect:data}



%-------------------------------------------------------------------------------

\section{Analysis and Results}
\label{sect:results}



%-------------------------------------------------------------------------------

\section{Conclusions}
\label{sect:concl}

From our simple analysis we draw the following conclusions:

\begin{enumerate}

\item The paintings done by citizen scientists span ... 

\item The mean (stacked) image and corresponding uncertainty map ... 

\end{enumerate}

%-----------------------------------------------------------------------

\section*{Acknowledgments} 

Painting on images for feature identification 
was conceived in discussions with the Zooniverse team in Oxford,
including Chris Lintott, Aprajita Verma, Arfon Smith and Rob Simpson. The
AstroTaches project was begun at the Dot Astronomy 3 meeting; we are grateful
to New College, Oxford, for providing the venue, and the conference organisers
for allowing us to hack the project together in an afternoon. 
%
% PJM received support from the Royal Society in the form of a research
% fellowship.                

%-------------------------------------------------------------------------------
\label{lastpage}

% \bibliography{references}

% MNRAS can be tricked into accepting bibtex but I forget how... 
% I used bubble to make a bbl file from our bib file:

% bubble -f astrotaches.tex references.bib 

% ==============================================================================

\documentclass[useAMS,usenatbib]{mn2e}

\usepackage[dvips]{graphicx}   % This is necessary to be able to include graphic
\usepackage{times}

\usepackage{epsfig}
\usepackage{amsmath,amssymb}
\usepackage{latexsym}
\usepackage{graphicx}% Include figure files
\usepackage{dcolumn}% Align table columns on decimal point
\usepackage{bm}% bold math
%\usepackage{epstopdf}

\def\reff@jnl#1{{\rm#1\/}}
\def\aj{\reff@jnl{AJ}}                  % Astronomical Journal
\def\araa{\reff@jnl{ARA\&A}}            % Annual Review of Astron and Astrophys
\def\apj{\reff@jnl{ApJ}}                % Astrophysical Journal
\def\apjl{\reff@jnl{ApJ}}               % Astrophysical Journal, Letters
\def\apjs{\reff@jnl{ApJS}}              % Astrophysical Journal, Supplement
\def\ao{\reff@jnl{Appl.Optics}}         % Applied Optics
\def\apss{\reff@jnl{Ap\&SS}}            % Astrophysics and Space Science
\def\aap{\reff@jnl{A\&A}}               % Astronomy and Astrophysics
\def\aapr{\reff@jnl{A\&A~Rev.}}         % Astronomy and Astrophysics Reviews
\def\aaps{\reff@jnl{A\&AS}}             % Astronomy and Astrophysics, Supplement
\def\azh{\reff@jnl{AZh}}                % Astronomicheskii Zhurnal
\def\baas{\reff@jnl{BAAS}}              % Bulletin of the AAS
\def\jrasc{\reff@jnl{JRASC}}            % Journal of the RAS of Canada
\def\memras{\reff@jnl{MmRAS}}           % Memoirs of the RAS
\def\mnras{\reff@jnl{MNRAS}}            % Monthly Notices of the RAS
\def\pra{\reff@jnl{Phys.Rev.A}}         % Physical Review A: General Physics
\def\prb{\reff@jnl{Phys.Rev.B}}         % Physical Review B: Solid State
\def\prc{\reff@jnl{Phys.Rev.C}}         % Physical Review C
\def\prd{\reff@jnl{Phys.Rev.D}}         % Physical Review D
\def\prl{\reff@jnl{Phys.Rev.Lett}}      % Physical Review Letters
\def\pasp{\reff@jnl{PASP}}              % Publications of the ASP
\def\pasj{\reff@jnl{PASJ}}              % Publications of the ASJ
\def\qjras{\reff@jnl{QJRAS}}            % Quarterly Journal of the RAS
\def\skytel{\reff@jnl{S\&T}}            % Sky and Telescope
\def\solphys{\reff@jnl{Solar~Phys.}}    % Solar Physics
\def\sovast{\reff@jnl{Soviet~Ast.}}     % Soviet Astronomy
\def\ssr{\reff@jnl{Space~Sci.Rev.}}     % Space Science Reviews
\def\zap{\reff@jnl{ZAp}}                % Zeitschrift fuer Astrophysik
\def\nat{\reff@jnl{Nature}}             % Nature 
\newcommand{\be}{\begin{equation}}
\newcommand{\ee}{\end{equation}}
\newcommand{\bea}{\begin{eqnarray}}
\newcommand{\eea}{\end{eqnarray}}
\newcommand{\bi}{\begin{itemize}}
\newcommand{\ei}{\end{itemize}}
% \newcommand\farcs{\mbox{$.\!\!^{\prime\prime}$}}
\def\vtr#1{{\bf #1}}
\def\mtx#1{{\bf #1}}

% ==============================================================================

\title%
[Astronomical Image Annotation]%
{AstroTaches: How Useful Are Annotative Paintings of Astronomical Images?}
\label{firstpage}

\author%
[Marshall et al]%
{Phil Marshall$^{1}$\thanks{E-mail:pjm@physics.ucsb.edu}, 
David W. Hogg$^{2}$,
Stuart R. Lowe$^{3}$
\newauthor Pamela L. Gay$^{4,5}$\\
$^{1}$Physics department, University of Oxford, \\
$^{2}$CCPP, NYU, New York, USA\\
$^{3}$Cardiff University, UK\\
$^{4}$Middle Illinois, USA\\
$^{5}$AstroSphere, USA}

\date{today}

\pagerange{\pageref{firstpage}--\pageref{lastpage}}

\pubyear{2011}

% ===============================================================================

\begin{document}

\maketitle

%-------------------------------------------------------------------------------

\begin{abstract}



\end{abstract}


\begin{keywords}
gravitational lensing --- surveys --- cosmology: observations
\end{keywords}


%-------------------------------------------------------------------------------

\section{Introduction}


%-------------------------------------------------------------------------------

\section{The AstroTaches Interface}
\label{sect:interface}



%-------------------------------------------------------------------------------

\section{A Simple Test Case: Gravitational Lens \lens}
\label{sect:testcase}


\begin{figure*}
\begin{center}
\includegraphics[width=0.9\linewidth]{figs/.eps} 
\caption{Predicted number density of lensing events $N$ per square degree as a function of the instrument angular resolution $\theta$/arcsec,
for the fixed values of 
    $\G=10^{-6}$ (dashed), 
       $10^{-7}$ (dotted) and
$5\times10^{-8}$ (solid).}\label{fig:bound}
\end{center}
\end{figure*}


%-------------------------------------------------------------------------------

\section{Paintings Generated by Zooniverse Users}
\label{sect:data}



%-------------------------------------------------------------------------------

\section{Analysis and Results}
\label{sect:results}



%-------------------------------------------------------------------------------

\section{Conclusions}
\label{sect:concl}

From our simple analysis we draw the following conclusions:

\begin{enumerate}

\item The paintings done by citizen scientists span ... 

\item The mean (stacked) image and corresponding uncertainty map ... 

\end{enumerate}

%-----------------------------------------------------------------------

\section*{Acknowledgments} 

Painting on images for feature identification 
was conceived in discussions with the Zooniverse team in Oxford,
including Chris Lintott, Aprajita Verma, Arfon Smith and Rob Simpson. The
AstroTaches project was begun at the Dot Astronomy 3 meeting; we are grateful
to New College, Oxford, for providing the venue, and the conference organisers
for allowing us to hack the project together in an afternoon. 
%
% PJM received support from the Royal Society in the form of a research
% fellowship.                

%-------------------------------------------------------------------------------
\label{lastpage}

% \bibliography{references}

% MNRAS can be tricked into accepting bibtex but I forget how... 
% I used bubble to make a bbl file from our bib file:

% bubble -f astrotaches.tex references.bib 

% ==============================================================================

\documentclass[useAMS,usenatbib]{mn2e}

\usepackage[dvips]{graphicx}   % This is necessary to be able to include graphic
\usepackage{times}

\usepackage{epsfig}
\usepackage{amsmath,amssymb}
\usepackage{latexsym}
\usepackage{graphicx}% Include figure files
\usepackage{dcolumn}% Align table columns on decimal point
\usepackage{bm}% bold math
%\usepackage{epstopdf}

\def\reff@jnl#1{{\rm#1\/}}
\def\aj{\reff@jnl{AJ}}                  % Astronomical Journal
\def\araa{\reff@jnl{ARA\&A}}            % Annual Review of Astron and Astrophys
\def\apj{\reff@jnl{ApJ}}                % Astrophysical Journal
\def\apjl{\reff@jnl{ApJ}}               % Astrophysical Journal, Letters
\def\apjs{\reff@jnl{ApJS}}              % Astrophysical Journal, Supplement
\def\ao{\reff@jnl{Appl.Optics}}         % Applied Optics
\def\apss{\reff@jnl{Ap\&SS}}            % Astrophysics and Space Science
\def\aap{\reff@jnl{A\&A}}               % Astronomy and Astrophysics
\def\aapr{\reff@jnl{A\&A~Rev.}}         % Astronomy and Astrophysics Reviews
\def\aaps{\reff@jnl{A\&AS}}             % Astronomy and Astrophysics, Supplement
\def\azh{\reff@jnl{AZh}}                % Astronomicheskii Zhurnal
\def\baas{\reff@jnl{BAAS}}              % Bulletin of the AAS
\def\jrasc{\reff@jnl{JRASC}}            % Journal of the RAS of Canada
\def\memras{\reff@jnl{MmRAS}}           % Memoirs of the RAS
\def\mnras{\reff@jnl{MNRAS}}            % Monthly Notices of the RAS
\def\pra{\reff@jnl{Phys.Rev.A}}         % Physical Review A: General Physics
\def\prb{\reff@jnl{Phys.Rev.B}}         % Physical Review B: Solid State
\def\prc{\reff@jnl{Phys.Rev.C}}         % Physical Review C
\def\prd{\reff@jnl{Phys.Rev.D}}         % Physical Review D
\def\prl{\reff@jnl{Phys.Rev.Lett}}      % Physical Review Letters
\def\pasp{\reff@jnl{PASP}}              % Publications of the ASP
\def\pasj{\reff@jnl{PASJ}}              % Publications of the ASJ
\def\qjras{\reff@jnl{QJRAS}}            % Quarterly Journal of the RAS
\def\skytel{\reff@jnl{S\&T}}            % Sky and Telescope
\def\solphys{\reff@jnl{Solar~Phys.}}    % Solar Physics
\def\sovast{\reff@jnl{Soviet~Ast.}}     % Soviet Astronomy
\def\ssr{\reff@jnl{Space~Sci.Rev.}}     % Space Science Reviews
\def\zap{\reff@jnl{ZAp}}                % Zeitschrift fuer Astrophysik
\def\nat{\reff@jnl{Nature}}             % Nature 
\newcommand{\be}{\begin{equation}}
\newcommand{\ee}{\end{equation}}
\newcommand{\bea}{\begin{eqnarray}}
\newcommand{\eea}{\end{eqnarray}}
\newcommand{\bi}{\begin{itemize}}
\newcommand{\ei}{\end{itemize}}
% \newcommand\farcs{\mbox{$.\!\!^{\prime\prime}$}}
\def\vtr#1{{\bf #1}}
\def\mtx#1{{\bf #1}}

% ==============================================================================

\title%
[Astronomical Image Annotation]%
{AstroTaches: How Useful Are Annotative Paintings of Astronomical Images?}
\label{firstpage}

\author%
[Marshall et al]%
{Phil Marshall$^{1}$\thanks{E-mail:pjm@physics.ucsb.edu}, 
David W. Hogg$^{2}$,
Stuart R. Lowe$^{3}$
\newauthor Pamela L. Gay$^{4,5}$\\
$^{1}$Physics department, University of Oxford, \\
$^{2}$CCPP, NYU, New York, USA\\
$^{3}$Cardiff University, UK\\
$^{4}$Middle Illinois, USA\\
$^{5}$AstroSphere, USA}

\date{today}

\pagerange{\pageref{firstpage}--\pageref{lastpage}}

\pubyear{2011}

% ===============================================================================

\begin{document}

\maketitle

%-------------------------------------------------------------------------------

\begin{abstract}



\end{abstract}


\begin{keywords}
gravitational lensing --- surveys --- cosmology: observations
\end{keywords}


%-------------------------------------------------------------------------------

\section{Introduction}


%-------------------------------------------------------------------------------

\section{The AstroTaches Interface}
\label{sect:interface}



%-------------------------------------------------------------------------------

\section{A Simple Test Case: Gravitational Lens \lens}
\label{sect:testcase}


\begin{figure*}
\begin{center}
\includegraphics[width=0.9\linewidth]{figs/.eps} 
\caption{Predicted number density of lensing events $N$ per square degree as a function of the instrument angular resolution $\theta$/arcsec,
for the fixed values of 
    $\G=10^{-6}$ (dashed), 
       $10^{-7}$ (dotted) and
$5\times10^{-8}$ (solid).}\label{fig:bound}
\end{center}
\end{figure*}


%-------------------------------------------------------------------------------

\section{Paintings Generated by Zooniverse Users}
\label{sect:data}



%-------------------------------------------------------------------------------

\section{Analysis and Results}
\label{sect:results}



%-------------------------------------------------------------------------------

\section{Conclusions}
\label{sect:concl}

From our simple analysis we draw the following conclusions:

\begin{enumerate}

\item The paintings done by citizen scientists span ... 

\item The mean (stacked) image and corresponding uncertainty map ... 

\end{enumerate}

%-----------------------------------------------------------------------

\section*{Acknowledgments} 

Painting on images for feature identification 
was conceived in discussions with the Zooniverse team in Oxford,
including Chris Lintott, Aprajita Verma, Arfon Smith and Rob Simpson. The
AstroTaches project was begun at the Dot Astronomy 3 meeting; we are grateful
to New College, Oxford, for providing the venue, and the conference organisers
for allowing us to hack the project together in an afternoon. 
%
% PJM received support from the Royal Society in the form of a research
% fellowship.                

%-------------------------------------------------------------------------------
\label{lastpage}

% \bibliography{references}

% MNRAS can be tricked into accepting bibtex but I forget how... 
% I used bubble to make a bbl file from our bib file:

% bubble -f astrotaches.tex references.bib 

\input{mn.bbl}

\bsp
%-------------------------------------------------------------------------------
\end{document}
% ==============================================================================


\bsp
%-------------------------------------------------------------------------------
\end{document}
% ==============================================================================


\bsp
%-------------------------------------------------------------------------------
\end{document}
% ==============================================================================


\bsp
%-------------------------------------------------------------------------------
\end{document}
% ==============================================================================
